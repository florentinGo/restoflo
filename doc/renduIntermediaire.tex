\documentclass[a4paper, 10pt, french]{article}
% Préambule; packages qui peuvent être utiles
\RequirePackage[T1]{fontenc}        % Ce package pourrit les pdf...
\RequirePackage{babel,indentfirst}  % Pour les césures correctes,
% et pour indenter au début de chaque paragraphe
\RequirePackage[utf8]{inputenc}   % Pour pouvoir utiliser directement les accents
% et autres caractères français
\RequirePackage{lmodern,tgpagella} % Police de caractères
\textwidth 17cm \textheight 25cm \oddsidemargin -0.24cm % Définition taille de la page
\evensidemargin -1.24cm \topskip 0cm \headheight -1.5cm % Définition des marges
\RequirePackage{latexsym}                  % Symboles
\RequirePackage{amsmath}                   % Symboles mathématiques
\RequirePackage{tikz}   % Pour faire des schémas
\RequirePackage{graphicx} % Pour inclure des images
\RequirePackage{listings} % pour mettre des listings
% Fin Préambule; package qui peuvent être utiles

\title{Rendu intermédiaire}
\author{
    ANOUFA Guillaume
    \\ GONDOIS Pierre
    \\ GONTHIER Florentin
    \\ LUPERINI Paul
    \\ MAHIEU Lucas
    \\ DE VALON Hugues
}

\begin{document}

\maketitle

%%%%%%%%%%%%%%%%%%%%%%%%%%%%%%%%%%%%%%%%%%%%%% 
\section{Schéma entités-associations}
{
    \begin{figure}[ht!]
        \includegraphics[width=\textwidth]{SchemaEntiteAssociation.png}
        \centering
        \caption{Schéma entités-association}
    \end{figure}
}

%%%%%%%%%%%%%%%%%%%%%%%%%%%%%%%%%%%%%%%%%%%%%% 
\section{Contraintes de valeurs}
{
    \begin{itemize}
        \item $nombrePlaceIsolée \geq nombrePlaceAccolée1 \geq nombrePlaceAccolée2 > 0$
        \item $numeroTable > 0$
        \item $prixArticle \geq 0$
        \item $type \in \{"entrée", "plat", "dessert", "boisson", "menu"\}$
        \item $typeService \in \{"midi", "soir"\}$
        \item $numeroClient > 0$
        \item $numeroReservation > 0$
        \item $nbPersonnes > 0$
        \item $prixTotal \geq 0$
    \end{itemize}
}
\section{Contraintes d'intégrité}
{
    \begin{itemize}
        \item Si le type d'article est un menu, il contient au moins un plat et un autre type d'article.
        \item Les articles composant un menu doivent être à 60\% de la même spécialité que le menu.
        \item Les tables d'un groupe ont la même localisation.
        \item Les articles commandés doivent apparaître sur la carte.
    \end{itemize}
}

%%%%%%%%%%%%%%%%%%%%%%%%%%%%%%%%%%%%%%%%%%%%%%
\end{document}
